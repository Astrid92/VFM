\subsection{Trayectorias}
  \subsubsection{CU1 Desplegar informaci\'on}
  \textbf{Resumen}
  \\
  \\
    El actor podr\'a visualizar la informaci\'on del itinerario elegido, esto es el mapa del transporte elegido, la descripci\'on de manera escrita
    de la ruta a seguir en el medio de transporte, el c\'alculo del presupuesto si as� lo requiere y un mapa de la ciudad con los lugares marcados
    de su itinerario para poder guiarse y llegar a pie.
  \\
  \\
  
  \textbf{Descripci\'on}
  \\
  \\
  \begin{tabular}{|p{3.5cm}|p{12cm}|l|}
     \hline
     \textbf{Caso de Uso:} & CU1 Desplegar informaci\'on \\ \hline
     \textbf{Actor:} & Turista\\ \hline
     \textbf{Prop\'osito:} & Visualizar la informaci�n del itinerario elegido\\ \hline
     \textbf{Entradas:} & -\\ \hline
     \textbf{Salidas: } & Por default muestra el mapa de la ruta del transporte.\\ \hline
     \textbf{Precondiciones: } & Un itinerario establecido.\\ \hline
     \textbf{Postcondiciones: } & - \\ \hline
     \textbf{Errores: } & -\\ \hline
     \textbf{Tipo:} & Prinario.\\ \hline
   \end{tabular}
   
   \subsubsection{Trayectoria Principal}
      \begin{enumerate}
	\item El \textbf{Actor} ingresa en la pantalla [VCU1.1].
	\item El \textbf{Sistema} carga la primer pesta�a de mostrar el mapa
	\item El \textbf{Sistema} muestra la pantalla de despliegue de informaci\'on con las siguientes opciones:
	\begin{itemize}
	 \item Calcular ruta.[PE CU1.1]
	 \item Mostrar informaci\'on del viaje. [PE CU1.2]
	 \item Mostrar mapa de la ciudad. [PE CU1.3]
	\end{itemize}
      \end{enumerate}	   
      --Fin del caso de uso.
      
\newpage
  \subsubsection{CU1.1 Calcular ruta}
  \textbf{Resumen}
  \\
  \\
    El sistema toma las coordenadas del turista y del destino base, por medio del GPS para calcular la ruta que llevar\'a al(los) turista(s)
    a la estaci\'on m\'as cercana del primer lugar de su itinerario, utilizando ya sea el metro o metrobus, dependindo de la elecci\'on del turista.
  \\
  \\
  
  \textbf{Descripci\'on}
  \\
  \\
  \begin{tabular}{|p{3.5cm}|p{12cm}|l|}
     \hline
     \textbf{Caso de Uso:} & CU1.1 Calcular ruta \\ \hline
     \textbf{Actor:} & Turista\\ \hline
     \textbf{Prop\'osito:} & Calcular la ruta para llegar al destino base.\\ \hline
     \textbf{Entradas:} & Las coordenadas del destino base, ubicaci\'on del turista, mediante el GPS
     y el medio de transporte a utilizar, seleccion\'andolo en un combobox\\ \hline
     \textbf{Salidas: } & El mapa del transporte seleccionado, con la ruta dibujada, que seguir\'a el
     usuario para llegar al destino base.\\ \hline
     \textbf{Precondiciones: } & El GPS del celular est\'a activado; La conexi\'on internet est\'a funcionando. Debe existir al menos un lugar destino y un
     cat\'alogo de transportes a utilizar.\\ \hline
     \textbf{Postcondiciones: } & - \\ \hline
     \textbf{Errores: } & [MSG4] Fallo en la conexi\'on de la Base de Datos. [MSG1] Informaci\'on incompleta.\\ \hline
     \textbf{Tipo:} & Secundario.\\ \hline
   \end{tabular}
   
   \subsubsection{Trayectoria Principal}
      \begin{enumerate}
	\item El \textbf{Actor} ingresa en la pantalla [VCU2].
	\item El \textbf{Actor} selecciona al menos un lugar a visitar.
	\item El \textbf{Actor} selecciona del combobox el medio de transporte que utilizar\'a.
	\item El \textbf{Actor} pulsa el bot\'on \textbf{Siguiente}.
	\item El \textbf{Sistema} valida que exista por lo menos un lugar a visitar.[Trayectoria Alternativa A]
	\item El \textbf{Sistema} calcula la ruta seg\'un el transporte seleccionado y el detino base. [Trayectoria Alternativa B]
	\item El \textbf{Sistema} muestra el mapa del transporte elegido, con la ruta a seguir para llegar
	a la estaci\'on m\'as cercana del destino base de la vista [VCU1.1].
      \end{enumerate}	   
      --Fin del caso de uso.
    
    \subsubsection{Trayectoria Alternativa A}
	\textbf{Condici\'on: } El turista no seleccion\'o por lo menos un lugar a visitar.
	 \begin{enumerate}
	      \item El \textbf{Actor} pulsa en el bot\'on \textbf{Siguiente}.
	      \item El \textbf{Sistema} valida que exista por lo menos un lugar a visitar.
	      \item El \textbf{Sistema} muestra el mensaje de error [MSG1] Informaci\'on incompleta,
	      mostrando el texto ``Falta llenar el campo Itinerario. Es obligatorio.''.
	      \item El \textbf{Actor} pulsa el bot\'on \textbf{Aceptar} del mensaje.
	      \item El \textbf{Sistema} regresa al paso 2 de la trayectoria principal.
	  \end{enumerate}
    --Fin de la trayectoria.
    
    \subsubsection{Trayectoria Alternativa B}
	\textbf{Condici\'on: } El sistema tuvo alg\'un problema con la conexi\'on de la base de datos.
	 \begin{enumerate}
	      \item El \textbf{Actor} hace clic en el bot\'on \textbf{Siguiente}.
	      \item El \textbf{Sistema} detecta un error en la conexi\'on de la base de datos.
	      \item El \textbf{Sistema} muestra el mensaje de error [MSG4] ``Error en la conexi\'on de la base de datos.''.
	      \item El \textbf{Actor} pulsa el bot\'on \textbf{Aceptar} del mensaje.
	      \item El \textbf{Sistema} regresa al paso 1 de la trayectoria principal.
	  \end{enumerate}
    --Fin de la trayectoria.
  
\newpage
    \subsubsection{CU1.2 Mostrar informaci\'on del viaje}
  \textbf{Resumen}
  \\
  \\
      El sistema calcular\'a el monto aproximado que se requiere para transportarse y visitar cada uno de los lugares en el itinerario
      selecionado, as\'i como mostrar de forma escrita la ruta que seguir\'a el turista en un medio de transporte seleccionado.
  \\
  \\
  
  \textbf{Descripci\'on}
  \\ 
  \\
  \begin{tabular}{|p{3.5cm}|p{12cm}|l|}
     \hline
     \textbf{Caso de Uso:} & CU1.2 Mostrar informaci\'on del viaje \\ \hline
     \textbf{Actor:} & Turista\\ \hline
     \textbf{Prop\'osito:} & Calcular el presupuesto del itinerario y mostrar la ruta de forma escrita.\\ \hline
     \textbf{Entradas:} & El n\'umero de personas ingresado en un campo de texto.\\ \hline
     \textbf{Salidas: } & El presupuesto total aproximado y la ruta de manera escrita.\\ \hline
     \textbf{Precondiciones: } & Un itinerario con los lugares a visitar (museos o restaurantes), la lista de precios de los lugares, as�
     como del transporte a utilizar. \\ \hline
     \textbf{Postcondiciones: } & - \\ \hline
     \textbf{Errores: } & [MSG4] Fallo en la conexi\'on de la Base de Datos. [MSG1] Informaci\'on incompleta.\\ \hline
     \textbf{Tipo:} & Secundario.\\ \hline
   \end{tabular}
   
   \subsubsection{Trayectoria Principal}
      \begin{enumerate}
	\item El \textbf{Actor} ingresa en la pantalla [VCU1.2].
	\item El \textbf{Sistema} despliega la ruta a seguir de forma escrita.
	\item El \textbf{Actor} ingresa el n\'umero de personas para realizar el c\'alculo
	del presupuesto del itinerario elegido.
	\item El \textbf{Actor} pulsa el bot\'on \textbf{Calcular presupuesto}.[Trayectoria Alternativa A]
	\item El \textbf{Sistema} realiza el calculo del presupuesto para el n\'umero de personas
	ingresado.[Trayectoria Alternativa B]
	\item El \textbf{Sistema} muestra el presupuesto total y por persona junto a la descripci\'on
	de la ruta del itinerario.

      \end{enumerate}	   
      --Fin del caso de uso.
    
    \subsubsection{Trayectoria Alternativa A}
	\textbf{Condici\'on: } El turista no ingres\'o el n\'umero de personas en el campo de texto
	para calcular el presupuesto.
	 \begin{enumerate}
	      \item El \textbf{Actor} pulsa en el bot\'on \textbf{Calcular presupuesto}.
	      \item El \textbf{Sistema} valida que se haya ingresado el n\'umero de personas para el presupuesto.
	      \item El \textbf{Sistema} muestra el mensaje de error [MSG1] Informaci\'on incompleta,
	      mostrando el texto ``Falta llenar el campo N\'umero de personas. Es obligatorio.''.
	      \item El \textbf{Actor} pulsa el bot\'on \textbf{Aceptar} del mensaje.
	      \item El \textbf{Sistema} regresa al paso 2 de la trayectoria principal.
	  \end{enumerate}
    --Fin de la trayectoria.
    
    \subsubsection{Trayectoria Alternativa B}
	\textbf{Condici\'on: } El sistema tuvo alg\'un problema con la conexi\'on de la base de datos.
	 \begin{enumerate}
	      \item El \textbf{Actor} hace clic en el bot\'on \textbf{Calcular presupuesto}.
	      \item El \textbf{Sistema} detecta un error en la conexi\'on de la base de datos.
	      \item El \textbf{Sistema} muestra el mensaje de error [MSG4] ``Error en la conexi\'on de la base de datos.''.
	      \item El \textbf{Actor} pulsa el bot\'on \textbf{Aceptar} del mensaje.
	      \item El \textbf{Sistema} regresa al paso 1 de la trayectoria principal.
	  \end{enumerate}
    --Fin de la trayectoria.
  
 \newpage
    \subsubsection{CU1.3 Mostrar mapa de la ciudad}
  \textbf{Resumen}
  \\
  \\
      El turista podr\'a visualizar el mapa de la ciudad, con los lugares de su itinerario marcados para poder tener una mejor
      referncia de como llegar y que calles se encuentran a su alrededor.
  \\
  \\
  
  \textbf{Descripci\'on}
  \\ 
  \\
  \begin{tabular}{|p{3.5cm}|p{12cm}|l|}
     \hline
     \textbf{Caso de Uso:} & CU1.3 Mostrar mapa de la ciudad \\ \hline
     \textbf{Actor:} & Turista\\ \hline
     \textbf{Prop\'osito:} & Visualizar el mapa de la ciudad con los lugares del itinerario.\\ \hline
     \textbf{Entradas:} & -\\ \hline
     \textbf{Salidas: } & El mapa con los lugares marcados\\ \hline
     \textbf{Precondiciones: } & Un itinerario previamente hecho. \\ \hline
     \textbf{Postcondiciones: } & - \\ \hline
     \textbf{Errores: } & [MSG8] Fallo en la conex\'on con google\\ \hline
     \textbf{Tipo:} & Secundario.\\ \hline
   \end{tabular}
   
    \subsubsection{Trayectoria Principal}
      \begin{enumerate}
	\item El \textbf{Actor} ingresa en la pantalla [VCU1.3] para visualizar el mapa de google.
	\item El \textbf{Sistema} se conecta con la API de google para mostrar el mapa.[Trayectoria Aleternativa A]
	\item El \textbf{Sistema} manda las coordenadas de los lugares para ubicarlas en el mapa.
	\item El \textbf{Sistema} despliega el mapa de la ciudad con los lugares ubicados.
      \end{enumerate}	   
      --Fin del caso de uso.
      \subsubsection{Trayectoria Alternativa A}
	\textbf{Condici\'on: } El sistema tuvo alg\'un problema con la conexi\'on de google.
	 \begin{enumerate}
	      \item El \textbf{Actor} hace clic en el bot\'on \textbf{Siguiente}.
	      \item El \textbf{Sistema} detecta un error en la conexi\'on de la base de datos.
	      \item El \textbf{Sistema} muestra el mensaje de error [MSG8] ``No se ha podido establecer conexi\'on.''.
	      \item El \textbf{Actor} pulsa el bot\'on \textbf{Aceptar} del mensaje.
	      \item El \textbf{Sistema} regresa al paso 1 de la trayectoria principal.
	  \end{enumerate}
    --Fin de la trayectoria.
    
    
  \newpage
  \subsubsection{CU2 Administrar itinerario}
  \textbf{Resumen}
  \\
  \\
  Le permitir\'a al usuario establecer el destino base y un radio m\'aximo de b\'usqueda de lugares cercanos
  para comenzar a crear su itinerario, a partir del destino base ingresado.
  \\
  \\
  
  \textbf{Descripci\'on}
  \\
  \\
  \begin{tabular}{|p{3.5cm}|p{12cm}|l|}
     \hline
     \textbf{Caso de Uso:} & CU2 Administrar itinerario \\ \hline
     \textbf{Actor:} & Turista\\ \hline
     \textbf{Prop\'osito:} & Establecer el destino base y el radio m\'aximo de b\'usqueda. \\ \hline
     \textbf{Entradas:} &  El turista ingresar\'a en un campo de texto el destino base, y en un numpicker
     el radio \'aximo para buscar los dem\'as lugares.\\ \hline
     \textbf{Salidas: } & Lista de los lugares posibles a visitar.\\ \hline
     \textbf{Precondiciones: } & Debe existir un cat\'alogo de lugares en la base de datos.\\ \hline
     \textbf{Postcondiciones: } & - \\ \hline
     \textbf{Errores: } & [MSG4] Fallo en la conexi\'on de la Base de Datos. [MSG1] Informaci\'on incompleta. [MSG7] Idioma no disponible.\\ \hline
     \textbf{Tipo:} & Primario.\\ \hline
   \end{tabular}
   
   \subsubsection{Trayectoria Principal}
      \begin{enumerate}
	\item El \textbf{Actor} ingresa en la pantalla [Vista Principal] para acceder al sistema.
	\item El \textbf{Sistema} valida el idioma del dispositivo m\'ovil respecto a la regla de negocio.
	[RN7].[Trayectoria Alternativa A]
	\item El \textbf{Actor} escribe en el campo de texto el destino base.
	\item El \textbf{Actor} establece un radio m\'aximo de b\'usqueda de lugares al rededor del destino base.
	\item El \textbf{Actor} pulsa el bot\'on de \textbf{Siguiente}.
	\item El \textbf{Sistema} valida la informaci\'on ingresada al sistema[Trayectoria alternativa B]
	\item El \textbf{Sistema} busca los lugares m\'as cercanos dependiendo del radio establecido. [Trayectoria Alternativa C]
      \end{enumerate}	   
      --Fin del caso de uso.
    
     \subsubsection{Trayectoria Alternativa A}
	\textbf{Condici\'on: } El sistema no encuentra el idioma del dispositivo m\'ovil en los disponibles para el sistema.
	 \begin{enumerate}
	      \item El \textbf{Actor} ingrsa a la pantalla [Vista Principal].
	      \item El \textbf{Sistema} valida el idioma del dispositivo m\'ovil.
	      \item El \textbf{Sistema} muestra el mensaje de error [MSG7] Idioma disponible.
	      \item El \textbf{Actor} pulsa el bot\'on \textbf{Aceptar} del mensaje.
	      \item El \textbf{Sistema} establece el idioma ingl\'es por default.
	      \item El \textbf{Sistema} regresa al paso 1 de la trayectoria principal.
	  \end{enumerate}
    --Fin de la trayectoria.
    
    \subsubsection{Trayectoria Alternativa B}
	\textbf{Condici\'on: } El turista no escribi\'o el destino base para realizar la b\'usqueda de los dem\'as lugares.
	 \begin{enumerate}
	      \item El \textbf{Actor} pulsa en el bot\'on \textbf{Siguiente}.
	      \item El \textbf{Sistema} valida que est\'e escrito el lugar destino en el campo de texto.
	      \item El \textbf{Sistema} muestra el mensaje de error [MSG1] Informaci\'on incompleta,
	      mostrando el texto ``Falta llenar el campo N\'umero de personas. Es obligatorio.''.
	      \item El \textbf{Actor} pulsa el bot\'on \textbf{Aceptar} del mensaje.
	      \item El \textbf{Sistema} regresa al paso 2 de la trayectoria principal.
	  \end{enumerate}
    --Fin de la trayectoria.
    
    \subsubsection{Trayectoria Alternativa C}
	\textbf{Condici\'on: } El sistema tuvo alg\'un problema con la conexi\'on de la base de datos.
	 \begin{enumerate}
	      \item El \textbf{Actor} hace clic en el bot\'on \textbf{Siguiente}.
	      \item El \textbf{Sistema} detecta un error en la conexi\'on de la base de datos.
	      \item El \textbf{Sistema} muestra el mensaje de error [MSG4] ``Error en la conexi\'on de la base de datos.''.
	      \item El \textbf{Actor} pulsa el bot\'on \textbf{Aceptar} del mensaje.
	      \item El \textbf{Sistema} regresa al paso 1 de la trayectoria principal.
	  \end{enumerate}
    --Fin de la trayectoria.
  
  \newpage
  \subsubsection{CU2.1 Agregar un lugar}
  \textbf{Resumen}
  \\
  \\
  Le permitir\'a al turista agregar a su itinerario un lugar tur\'istico o restaurante. El cual puedr\'a ser el punto
  de inicio para calcular la ruta de llegada.
  \\
  \\
  
  \textbf{Descripci\'on}
  \\
  \\
  \begin{tabular}{|p{3.5cm}|p{12cm}|l|}
     \hline
     \textbf{Caso de Uso:} & CU2.1 Agregar un lugar \\ \hline
     \textbf{Actor:} & Turista\\ \hline
     \textbf{Prop\'osito:} & Seleccionar lugares tur\'isticos o restaurante a un itinerario. \\ \hline
     \textbf{Entradas:} & Mediante un \textbf{checkbox} se crear\'a la lista de los lugares tur\'isticos
     y/o restaurantes a visitar por el turista.\\ \hline
     \textbf{Salidas: } & Listado de los lugares seleccionados. \\ \hline
     \textbf{Precondiciones: } & Debe existir el cat\'alogo de lugares tur\'isticos y restaurantes.\\ \hline
     \textbf{Postcondiciones: } & - \\ \hline
     \textbf{Errores: } & [MSG4] Fallo en la conexi\'on de la Base de Datos.,  [MSG5] Limite de lugares permitido. y [MSG6] M\'inimo de lugares seleccionado \\ \hline
     \textbf{Tipo:} & Segundario.\\ \hline
   \end{tabular}
   
   \subsubsection{Trayectoria Principal}
      \begin{enumerate}
	\item El \textbf{Actor} ingresa en la pantalla [VCU2.1].[Trayectoria Alternativa A]
	\item El \textbf{Sistema} se conecta a la base de datos. [Trayectoria Alternativa B]
	\item El \textbf{Sistema} muestra una lista de los lugares posibles a seleccionar.
	\item El \textbf{Actor} selecciona 5 lugares que desee visitar. [PE CU2.2] [Trayectoria Alternativa C]
	\item El \textbf{Actor} pulsa el bot\'on \textbf{Siguiente}.
	\item El \textbf{Sistema} muestra el mensaje [MSG3] Confirmaci\'on ``?`Seguro que desea calcular la ruta?''.[Trayectoria Alternativa D]
	\item El \textbf{Actor} pulsa el bot\'on \textbf{Aceptar} para confirmar su decisi\'on.
	\item El \textbf{Sistema} muestra la pantalla [VCU1].
      \end{enumerate}	   
      --Fin del caso de uso.
      
    \subsubsection{Trayectoria Alternativa A}
	\textbf{Condici\'on: } El turista decidi\'o cambiar de lugar tur\'istico base.
	 \begin{enumerate}
	      \item El \textbf{Actor} hace clic en el bot\'on \textbf{Atr\'as}.
	      \item El \textbf{Sistema} regresa a va pantalla [Vista principal del sistema].
	  \end{enumerate}
    --Fin del caso de uso. 
    
    \subsubsection{Trayectoria Alternativa B}
	\textbf{Condici\'on: } El sistema tuvo alg\'un problema con la conexi\'on de la base de datos.
	 \begin{enumerate}
	      \item El \textbf{Actor} hace clic en el bot\'on \textbf{Siguiente}.
	      \item El \textbf{Sistema} detecta un error en la conexi\'on de la base de datos.
	      \item El \textbf{Sistema} muestra el mensaje de error [MSG4] ``Error en la conexi\'on de la base de datos.''.
	      \item El \textbf{Actor} pulsa el bot\'on \textbf{Aceptar} del mensaje.
	      \item El \textbf{Sistema} regresa al paso 1 de la trayectoria principal.
	  \end{enumerate}
    --Fin de la trayectoria.
    
    \subsubsection{Trayectoria Alternativa C}
	\textbf{Condici\'on: } El actor no seleccion\'o alg\'un lugar para visitar.
	 \begin{enumerate}
	      \item El \textbf{Actor} pulsa el bot\'on \textbf{Siguiente}.
	      \item El \textbf{Sistema} muestra el mensaje de error [MSG6] ``Error, debes seleccionar al menos 1 lugar.''.
	      \item El \textbf{Actor} pulsa el bot\'on \textbf{Aceptar}.
	      \item El \textbf{Sistema} regresa al paso 4 de la trayectoria principal.
	  \end{enumerate}
    --Fin de la trayectoria.
    
        \subsubsection{Trayectoria Alternativa D}
	\textbf{Condici\'on: } El actor decidi\'o cambiar su itinerario.
	 \begin{enumerate}
	      \item El \textbf{Actor} hace clic en el bot\'on \textbf{Cancelar}.
	      \item El \textbf{Sistema} regresa al paso n\'umero 4 de la trayectoria principal.
	  \end{enumerate}
    --Fin de la trayectoria.
  
 \newpage
 \subsubsection{CU2.2 Eliminar un lugar}
  \textbf{Resumen}
  \\
  \\
  El turista podr\'a decidir quitar un lugar del itinerario que ha creado.
  \\
  \\
  
  \textbf{Descripci\'on}
  \\
  \\
  \begin{tabular}{|p{3.5cm}|p{12cm}|l|}
     \hline
     \textbf{Caso de Uso:} & CU2.2 Eliminar un lugar \\ \hline
     \textbf{Actor:} & Turista\\ \hline
     \textbf{Prop\'osito:} & Quitar lugares tur\'isticos del itinerario. \\ \hline
     \textbf{Entradas:} & Un itinerario con por lo menos 1 lugar tur\'istico \\ \hline
     \textbf{Salidas: } & Una lista con menor n\'umero de lugares tur\'isticos en el itinerario. \\ \hline
     \textbf{Precondiciones: } & Debe haberse elegido por lo menos un lugar tur\'istico. \\ \hline
     \textbf{Postcondiciones: } & - \\ \hline
     \textbf{Errores: } & - \\ \hline
     \textbf{Tipo:} & Secundario.\\ \hline
   \end{tabular}
   
   \subsubsection{Trayectoria Principal}
      \begin{enumerate}
	\item El \textbf{Actor} ingresa en la pantalla [VCU2.2].[Trayectoria Alternativa A]
	\item El \textbf{Sistema} muestra la lista de los lugares seleccionados.
	\item El \textbf{Actor} pulsa el bot\'on con una \textbf{X} para quitar un lugar tur\'istico.
	\item El \textbf{Sistema} muestra un lugar tur\'istico menos en el itinerario.
      \end{enumerate}	   
      --Fin del caso de uso.
      
    \subsubsection{Trayectoria Alternativa A}
	\textbf{Condici\'on: } El turista decidi\'o cambiar de lugar tur\'istico base.
	 \begin{enumerate}
	      \item El \textbf{Actor} hace clic en el bot\'on \textbf{Atr\'as}.
	      \item El \textbf{Sistema} regresa a va pantalla [Vista Principal].
	  \end{enumerate}
    --Fin del caso de uso.

\newpage
  
  \subsubsection{CU3 Cambiar idioma}
  \textbf{Resumen}
  \\
  \\
	La informaci\'on contenida en la aplicaci\'on ser\'a mostrada al usuario en uno de los 3 idiomas disponibles:
	\begin{enumerate}
	  \item Espa\~nol
	  \item Ingles
	  \item Frances
	\end{enumerate}
  
  
  \textbf{Descripci\'on}
  \\
  \\
  \begin{tabular}{|p{3.5cm}|p{12cm}|l|}
     \hline
     \textbf{Caso de Uso:} & CU3 Cambiar idioma \\ \hline
     \textbf{Actor:} & -\\ \hline
     \textbf{Prop\'osito:} & Mostrar la informaci\'on de la aplicaci\'on como m\'as le agrade al usuario. \\ \hline
     \textbf{Entradas:} & El el idioma seleccionado por el usuario. \\ \hline
     \textbf{Salidas: } & El texto de informaci\'on de la aplicaci\'on se mostrar\'a en el idioma selccionado. \\ \hline
     \textbf{Precondiciones: } &  -\\ \hline
     \textbf{Postcondiciones: } & - \\ \hline
     \textbf{Errores: } & [MSG4] Fallo en la conexi\'on de la Base de Datos.\\ \hline
     \textbf{Tipo:} & Primario.\\ \hline
   \end{tabular}
   
   \subsubsection{Trayectoria Principal}
      \begin{enumerate}
	\item El \textbf{Actor} ingresa en la pantalla [VCU3].
	\item El \textbf{Actor} selecciona el idioma de su preferencia.
	\item El \textbf{Actor} pulsa el bot\'on \textbf{Cambiar idioma}{Trayectoria Aleternativa A}
	\item El \textbf{Sistema} configura el idioma seleccionado por el usuario.

      \end{enumerate}	   
      --Fin del caso de uso.
      
    \subsubsection{Trayectoria Alternativa A}
	\textbf{Condici\'on: } El sistema tuvo alg\'un problema con la conexi\'on de la base de datos.
	 \begin{enumerate}
	      \item El \textbf{Actor} hace clic en el bot\'on \textbf{Cambiar idioma}.
	      \item El \textbf{Sistema} detecta un error en la conexi\'on de la base de datos.
	      \item El \textbf{Sistema} muestra el mensaje de error [MSG4] ``Error en la conexi\'on de la base de datos.''.
	      \item El \textbf{Actor} pulsa el bot\'on \textbf{Aceptar} del mensaje.
	      \item El \textbf{Sistema} regresa al paso 1 de la trayectoria principal.
	  \end{enumerate}
    --Fin de la trayectoria.
  