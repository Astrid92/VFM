
\section{Introducci\'on}
\subsection{Presentaci\'on del documento}
En el presente documento tiene como objetivo presentar\'a el an\'alisis del desarrollo del sistema, abarcando los siguientes puntos:
\begin{itemize}
 \item \textbf{Modelo de despliegue}
 \item \textbf{Modelo del comportamiento}
 \item \textbf{Modelo de interfaces}
 \item \textbf{Modelo del dominio del problema} 
\end{itemize}

\subsection{Presentaci\'on del equipo}
 El equipo AndroidMex esta conformado por:
 \begin{itemize}
 	\item Ch\'avez Delgado Eduardo, que actualmente cursa el 7mo semestre en la ESCOM y actual administrador
 del \'area de sistemas en RACOM Microelectronics.
 	\item Yoan Tournade, quien es un estudiante franc\'es de la UTC, desarrollador free-lance por el sector de la seguridad de los edificios y contribuidor por proyectos inform\'aticos de asociaciones de estudiantes de su escuela francesa.
 	\item Viveros Pedraza Astrid Esperanza, quien cursa el 7mo semestre de la Ingenier\'ia en Sistemas Computacionales en la ESCOM del IPN ha colaborado en proyectos escolares y de investigaci\'on 
 en el programa PIFI.
 \end{itemize}
 
\subsection{Presentaci\'on del proyecto}
Actualmente el turismo en el Distrito Federal, no es promovido adecuadamente de manera local, nacional e incluso internacionalmente, 
adem\'as existen numerosos lugares que pocas personas conocen o se desconoce la ruta a seguir para poder llegar a dichos lugares. 
Por otra parte ocasionalmente la informaci\'on que llegan a presentar en su p\'agina web (si es que tienen), no es suficiente y no siempre son 
incluidos los horarios y precios del lugar. Finalmente tambi\'en es complicado realizar un peque\~no itinerario debido a esta falta de informaci\'on. 

\subsection{Alcance del proyecto}
Por ello se propone hacer un sistema utilizando tecnolog\'ias m\'oviles, para trazar una ruta de llegada a zonas tur\'isticas accesibles solo utilizando 
Metro y Metro Bus, en \'este trabajo, solo se enfocar\'a a 30 museos de la ciudad, adem\'as de generar un itinerario, que mostrar\'a tanto los museos a visitar
como los restaurantes VIPS y MCDonalds de esta zona, permitiendo al usiario modificar el itinerario antes de calcular la ruta hacia el destino base. Finalmente, se calcular\'a un
\textit{\textbf{aproximado}} del gasto al elegir dicho itinerario, para una o varias personas.
El sistema esta dirigido al p\'ublico de 13 a\~nos en adelante, con el objetivo de promover los sitios tur\'isticos del Distrito Federal
a trav\'es de las tecnolog\'ias m\'oviles.


